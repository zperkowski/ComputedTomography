\documentclass[12pt,a4paper]{article}
\usepackage[utf8]{inputenc}
\usepackage[polish]{babel}
\usepackage{polski}
\usepackage{pgfplots}
\usepackage{color}
\usepackage{listings}
\lstset{language=C}
\usepackage{hyperref}
\usepackage{csvsimple}
\usepackage{tikz}
\usetikzlibrary{matrix}
\usepackage{fullpage,amsmath}
\usepackage{mathtools}
\DeclarePairedDelimiter\ceil{\lceil}{\rceil}
\DeclarePairedDelimiter\floor{\lfloor}{\rfloor}
\usetikzlibrary{arrows,shapes}
\pgfplotsset{width=10cm}
\usepackage[left=1cm,right=1cm,top=2cm,bottom=2cm]{geometry}
\renewcommand{\thesection}{\Roman{section}} % zmiana numeracji sekcji na rzymskie
\lstset{language=Java,
    extendedchars=\true
}
% Wykresy
\usepackage{pgfplots}
\usepackage{SIunits}
\pgfplotsset{compat=newest}

% Tabele
 \usepackage{pgfplotstable}
 \usepackage{booktabs}

 \usepackage{longtable}
 \pgfplotstableset{
 begin table=\begin{longtable},
 end table=\end{longtable},
 }

\author{Zdzisław Perkowski \& Paweł Szajda}
\title{Computed Tomography}

\begin{document}

\begin{center}
{\Huge Informatyka w medycynie\\ Computed Tomography\\}
\begin{large}
Zdzisław Perkowski\\
Paweł Szajda
\end{large}
\end{center}
\section{Wstęp}

\section{Wydajność}

\begin{tikzpicture}
\begin{axis}[
	xlabel={Wielkość obrazu [pixele]},
	ylabel={Czas [ms]},
	title={Czas obliczeń},
	grid=both,
	legend style={at={(0.05,0.9)},anchor=south west},
	minor grid style={gray!25},
	major grid style={gray!25},
	width=1\linewidth,
	no marks]
%\addplot[line width=1pt,solid,color=blue]
%	table[x=Matrix,y=Time,col sep=space]{Dane/Cuda_Global_256_16.txt};
\addlegendentry{Czas [ms]};
\end{axis}
\end{tikzpicture}

\section{Dokładność}

\section{Wnioski}


\section{Kody źródłowe}
Pełna wersja oprogramowania znajduje się pod adresem: \\ \href{http://GitHub.com/zperkowski/ComputedTomography}{http://GitHub.com/zperkowski/ComputedTomography}.
\subsection{Interesujący fragment}
\begin{lstlisting}
int main(int argc, char const *argv[]) {
    return 0;
}
\end{lstlisting}
\end{document}
